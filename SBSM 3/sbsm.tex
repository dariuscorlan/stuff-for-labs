\documentclass[a4paper]{article}
\usepackage{graphicx}
\usepackage[margin = 1in]{geometry}
\usepackage{ragged2e}
\usepackage[hidelinks, colorlinks = true, citecolor = black, linkcolor = blue]{hyperref}
\usepackage{parskip}
\usepackage{cite}
\usepackage{caption}
\usepackage{subcaption}
\usepackage{cellspace}
\usepackage{makecell}
\usepackage{mwe}
\setlength\cellspacetoplimit{4pt}
\setlength\cellspacebottomlimit{4pt}
\usepackage{caption} 
\usepackage{pgfplots}
\usepackage{amsmath}
\usepackage{tikz}
\usepackage{pdfpages}
\newcommand{\inv}{^{\raisebox{.2ex}{$\scriptscriptstyle-1$}}}
\newcommand{\unit}[1]{~\mathrm{#1}}
\captionsetup[table]{skip=10pt}
\renewcommand{\arraystretch}{1.5}
\pgfplotsset{compat=1.16}
\pgfplotsset{ignore zero/.style={%
  #1ticklabel={\ifdim\tick pt=0pt \else\pgfmathprintnumber{\tick}\fi}
}}

\begin{document}
\section{Introduction}
The objective of this experiment is to determine and compare the hardness of
various metal samples using the Brinell, Vickers, and Rockwell methods. The
study focuses on evaluating the influence of material treatment such as
quenching and plastic deformation on hardness. Appropriate loads and indenters
are applied for each method, and multiple indentations are conducted to ensure
accuracy.

Results are analyzed using standardized conversion tables to establish correlations between hardness and tensile strength, as well as to assess the suitability of each testing method for different materials and applications.

\section{Background/ Theory}

The hardness of metals is measured via standardized tests, in which a small
indenter of a defined shape is pressed into the material with a certain force.
The hardness is then deduced from the dimensions of the indentation remaining
after the test (Vickers and Brinell methods). In the Rockwell methods, one reads
the hardness directly on the device on a dial gauge, which actually measures the
depth of this indentation.

\section{Method \& Materials}

A hardness apparatus is used for the test, in which one mounts the desired
indenter, and one adjusts the desired force. A pre-load of 10 kgf is manually
applied to the sample beforehand by turning a wheel, pushing the sample against
the indenter with a force of 10 kgf. The rest of the measurement is performed
automatically: via a mechanism, the main load is applied, held for a few
seconds, and then removed.

5 total samples tested using multiple types of hardness tests \textemdash~two
C-45 steel blocks, one of them being quenched, a brass strip, and two steel
tensile samples, one of them having undergone a tensile test. Firstly, the two
C-45 steel cubes and the brass strip are subjected to the Vickers hardness test.
Afterwards, all the samples get subjected to the two Rockwell tests. To
determine which rockwell test would be appropriate, the Vickers rating is used.
Finally, the brass sample is subjected to the Brinell test. 
\newpage

\section {Results}

\subsection{Measurements}
\subsubsection{Vickers measurements}
All of the samples were subjected to a load of $100\unit{kgf}$ and measured with a digital microscope which was able to digitally
measure the dimensions of the indentation. The results of these measurements is
displayed in table 1.

\begin{table}[!ht]
  \centering
  \label{tab:1}
  \caption{Measurement results of Vickers hardness test}
  \begin{tabular}{c|cc} 

  Material     & \makecell{$d_1$\\ $\unit{(mm)}$}    & \makecell{$d_2$\\$\unit{(mm)}$}     \\ 
  \hline
  $\mathrm{Steel_{hardened_1}}$   & 0.488 & 0.498  \\
  $\mathrm{Steel_{hardened_2}}$    & 0.5   & 0.515  \\
  $\mathrm{Steel_{hardened_3}}$    & 0.499 & 0.5    \\
  $\mathrm{Steel_{unhardened_1}}$ & 1.09  & 1.118  \\
  $\mathrm{Steel_{unhardened_2}}$  & 1.067 & 1.085  \\
  $\mathrm{Steel_{unhardened_3}}$  & 1.08  & 1.09   \\
  $\mathrm{Brass_1}$        & 1.06  & 1.054  \\
  $\mathrm{Brass_2}$        & 1.061 & 1.048  \\
  $\mathrm{Brass_3}$        & 1.6   & 1.58   \\

  \end{tabular}
  \end{table}

Afterwards, the necessary calculations are done to determine the Vickers
hardness, and the results are placed in table 2.

\begin{table}[!ht]
  \centering
  \label{tab:2}
  \caption{Hardness value results of the Vickers test}
  \begin{tabular}{c|cccc} 
  Material     & \makecell{$d_{avr}$\\$\unit{(mm)}$}    & \makecell{$HV_{avr}$ \\ $\unit{(-)}$}       & \makecell{$\Delta
  HV$\\ $\unit{(-)}$}   & \makecell{$HV_{standard ~notation}$ \\ $\unit{(-)}$}  \\ 
  \hline
  $\mathrm{Steel_{hardened}}$   & 0.5 & 741.33 & 200 & $(741.33 \pm 200)$   \\
  $\mathrm{Steel_{unhardened}}$ & 1.09    & 156.12 & 30 & $(156.12 \pm 30)$   \\
  $\mathrm{Brass}$        & 1.06 & 166.33 & 30 & $(166.33 \pm 30)$   \\

  \end{tabular}
  \end{table}

\subsubsection{Rockwell measurements}
By using the results of the Vickers hardness test, the kind of Rockwell test
that needed to be used was determined. For the hardened steel, Rockwell C was
used, whereas for the brass and the unhardened steel, Rockwell B was used.

\newpage
The results of the respective tests are arranged in table 3.

\begin{table}[!ht]
  \centering
  \label{tab:3}
  \caption{Rockwell hardness results for both of the steel blocks and brass strip}
  \begin{tabular}{c|ccc} 
  Measurement & \makecell{$\mathrm{Steel_{hardened}}$\\$\unit{(HRC)}$} &
  \makecell{$\mathrm{Steel_{unhardened}}$\\$\unit{(HRB)}$} & \makecell{$\mathrm{Brass}$\\ $\unit{(HRB)}$}  \\ 
  \hline
  1           & 67         & 80         & 86     \\
  2           & 62         & 82         & 86     \\
  3           & 63         & 81         & 87     \\
  4           & 61         & 83         & 87     \\
  5           & 83         & 84         & 88     \\
  \end{tabular}
  \end{table}

After taking the average of these values and calculating the error, the final
results are displayed in table 4.

\begin{table}[!ht]
  \centering
  \label{tab:4}
  \caption{Final result of the Rockwell hardness tests}
  \begin{tabular}{c|ccc}
  Material    & \makecell{$HR_{avr}$ \\ $\unit{(-)}$} & \makecell{$\Delta HR$ \\
  $\unit{(-)}$}  & \makecell{$HR_{standard~notation}$ \\ $\unit{(-)}$}  \\ 
  \hline
  $\mathrm{Steel_{hardened}}$   & 63.2  & 2 & $(63.2 \pm 2)$       \\
  $\mathrm{Steel_{unhardened}}$ & 82    & 1 & $(82 \pm 1)$       \\
  $\mathrm{Brass}$       & 86.8  & 1 & $(86.8 \pm 1)$      
  \end{tabular}
\end{table}

\subsubsection{Rockwell B measurements}
For both tensile samples, the Rockwell B hardness test was used. The
measurements are displayed in table 5.

\begin{table}[!ht]
  \centering
  \label{tab:5}
  \caption{All measurement results of the tensile samples}
  \begin{tabular}{c|cc}
  Measurement & \makecell{$\mathrm{Sample_{original}}$\\ $\unit{(-)}$} &
  \makecell{$\mathrm{Sample_{drawn}}$ \\ $\unit{(-)}$}  \\ 
  \hline
  1           & 65        & 82         \\
  2           & 61        & 86         \\
  3           & 69        & 87         \\
  4           & 58        & 86         \\
  5           & 61        & 87        
  \end{tabular}
\end{table}

Afterwards, the same steps to calculate the error are done as before, the final
results being displayed in table 6.

\begin{table}[!ht]
  \centering
  \label{tab:6}
  \caption{Final results of the tensile sample tests}
  \begin{tabular}{c|ccc}
  Material & \makecell{$HR_{avr}$\\$\unit{(-)}$} & \makecell{$\Delta HR$ \\
  $\unit{(-)}$}  & \makecell{$HR_{standard~notation}$ \\ $ \unit{(-)}$}  \\ 
  \hline
  $\mathrm{Sample_{drawn}}$ & 85.6  & 2 & $(85.6 \pm 2)$       \\
  $\mathrm{Sample_{original}}$ & 82    & 1 & $(82 \pm 1)$      
  \end{tabular}
\end{table}

\newpage
\subsubsection{Brinell for brass}
Firstly to utilize the Brinell hardness test, the main load must be calculated.
The force is determined to be $60\unit{kgf}$. Afterwards, the results of the
indentation measurements are noted down in table 7.

\begin{table}[!ht]
  \centering
  \label{tab:7}
  \caption{Indentation results of the Brinell test}
  \begin{tabular}{c|c}
  Measurement & \makecell{$d_{indentation}$\\ $\unit{(mm)}$}      \\ 
  \hline
  1           & 0.728  \\
  2           & 0.709  \\
  3           & 0.719  \\
  4           & 0.728  \\
  5           & 0.72  
  \end{tabular}
\end{table}

Finally, the final hardness value can be calculated. The results of this
calculation are shown in table 8.

\begin{table}[!ht]
  \centering
  \label{tab:8}
  \caption{Hardness measurement results of the Brinell test}
  \begin{tabular}{c|ccc}
  \makecell{$d_{avr}$ \\ $\unit{(mm)}$}   & \makecell{$HB_{avr}$ \\
  $\unit{(-)}$}  & \makecell{$\Delta HB$ \\ $\unit{(-)}$} &
  \makecell{$HB_{standard~notation}$ \\ $\unit{(-)}$}  \\ 
  \hline
  0.721 & 150 & 20     & $(150 \pm 20)$    
  \end{tabular}
\end{table}

\subsection{Calculations}
\end{document}