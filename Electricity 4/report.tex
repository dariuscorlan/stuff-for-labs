\documentclass[a4paper]{article}
\usepackage{graphicx}
\usepackage[margin = 1in]{geometry}
\usepackage{ragged2e}
\usepackage[hidelinks, colorlinks = true, citecolor = black, linkcolor = blue]{hyperref}
\usepackage{parskip}
\usepackage{cite}
\usepackage{caption}
\usepackage{subcaption}
\usepackage{cellspace}
\usepackage{makecell}
\usepackage{mwe}
\setlength\cellspacetoplimit{4pt}
\setlength\cellspacebottomlimit{4pt}
\usepackage{caption} 
\usepackage{pgfplots}
\usepackage{amsmath}
\usepackage{tikz}
\usepackage{pdfpages}
\newcommand{\inv}{^{\raisebox{.2ex}{$\scriptscriptstyle-1$}}}
\newcommand{\unit}[1]{~\mathrm{#1}}
\captionsetup[table]{skip=10pt}
\renewcommand{\arraystretch}{1.5}

\begin{document}
\section{Introduction}
DC power sources are an essential part of circuits. The aim of the following
experiments is to study the proprieties of these sources. Electromotive force,
ideal-source current, open-circuit voltage, short-circuit current, and internal
resistance will be studied for voltage sources.

The behaviour of series and parallel source arrangements will also be studied.

\section{Background/ Theory}
DC power sources refer to either voltage or current sources of which the output
does not diminish with time. A source is called an ideal voltage source when the
output voltage does not change in relation to the load resistance. For current
sources, the current remains constant regardless of the load resistor. 

In physical applications however, the current or voltage will depend on the load
resistance. Therefore, power sources will be represented as a combination
between an ideal source and an internal resistance. For voltage sources, the
resistance will be connected in series with the ideal source, whereas for
current sources, the resistor will be connected in parallel. (book)

Figures 1(a) and 1(b) display the aforementioned representation of power
sources, both for current and voltage sources.(lab text)

\begin{figure}[!ht]
    \centering
    \begin{subfigure}{0.5\textwidth}
        \centering
        \includegraphics[width = 0.5 \linewidth]{Voltagesource.png}
        \label{fig:1a}
        \caption{Schematic representation of voltage source}
    \end{subfigure}%
    \begin{subfigure}{0.5\textwidth}
        \centering
        \includegraphics[width = 0.7\linewidth]{Currentsource.png}        
        \label{fig:1b}
        \caption{Schematic representation of current source}
    \end{subfigure}
    \label{fig:1}
    \caption{Schematic representation of power sources}  
\end{figure}

\subsection{Source characteristics}
Using source transformation, all power sources can be represented as voltage
sources with an internal resistance(book). Therefore, going forward, sources
will be displayed as shown in Figure 1(a). This representation is called the
voltage-equivalent circuit. By adding a load resistance to the power source, essential characteristics of
the source can be discovered.

\newpage
Figure 2 displays the circuit which will be used to determine these
characteristics(labtext).

\begin{figure}[!ht]
    \centering
    \includegraphics[width = 0.5\textwidth]{loadcircuit.png}
    \label{fig:2}
    \caption{Simple load circuit of real voltage source}
\end{figure}

Firstly, the voltage of the ideal source is called the electromotive force,
$V_{emf}$. By measuring the load current, this voltage can be related to the
load voltage by observing that the same current passes through the internal
resistance as well. 

Equation 1 displays the relationship between current, load
voltage, and electromotive force(labtext).

\begin{equation}
    V_L = V_{emf} + I_L R_i
\end{equation}

Since the internal resistance, $R_i$, is constant, the relationship between load
voltage and current is a linear one. 

Figure 3 displays this relationship for a
real voltage source such as the one shown in Figure 1(a)(labtext).

\begin{figure}[!ht]
    \centering
    \includegraphics[width = 0.5\textwidth]{vi_graph.png}
    \label{fig:3}
    \caption{Typical $V_L$ - $I_L$ graph used to determine source characteristics}
\end{figure}

By determining the slope of the linear relationship between $I_L$ and $V_L$, the
internal resistance can be found(labtext):

\begin{equation}
    R_i = -\frac{\Delta V_L}{\Delta I_L}
\end{equation}

By looking at the x and y intercepts of the graph, both the electromotive force
and the short circuit current can be found. Therefore, all of the sources
characteristics can be found by analyzing the Voltage-Current graph.

A relationship between all of the characteristics is found equation 3(labtext):

\begin{equation}
    I_{sc} = \frac{V_{emf}}{R_i}
\end{equation}

\subsection{Series and parallel sources}
By connecting multiple voltage sources in series, higher $V_{emf}$ values can be
provided. The equivalent internal resistance will be equal to the sum of the
individual sources internal resistances.

When connecting multiple sources in parallel, the equivalent internal resistance
will be lowered. The lowering of the internal resistance will be governed by the
parallel resistance equivalence(labtext):

\begin{equation}
    R_i = \left( \sum_{j} \frac{1}{R_{i,~j}}\right) \inv
\end{equation}

Since sources should only be connected in parallel when they have the same
electromotive force, the equivalent $V_{emf}$ will be equal to that of one
source(labtext).

\subsection{Maximum power transfer}

\section{Method and materials}
All circuits in the experiments were assembled using the connection boards. All
measurements, including voltage and current, were determined using an AM-520
HVAC mutlimeter. 1.5 V batteries were used as voltage sources. Both a decade
resistor, and two fixed resistors of $62\unit{\Omega}$ and $150\unit{\Omega}$.

Throughout the experiments, both the load current, $I_L$, and load voltage,
$V_L$, are measured at every itteration.

\subsection{Experimental Set-up determination of internal resistance}
The set-up for the first experiment is the same as the one described in Figure
2, where two different fixed resistors are used for $R_i$. The value of the
resistors are $62\unit{\Omega}$ and $150\unit{\Omega}$. A decade resistor is
used to vary the load resistor in order to complete the necessary measurements.

To determine the source characteristics, the following experimental flow is
established. Firstly the open circuit voltage is determined, thus establishing
$V_{emf}$. Afterwards the decade resistor's value is adjusted such that a
certain load current is established. The value of $I_L$ varies between
$0.5\unit{mA}$ and $5\unit{mA}$ in increments of $0.5\unit{mA}$. After each
adjustment, both the load voltage and the resistance of the decade resistor is
noted down.

Figure 4 displays the physical experimental set-up used to gather the
measurements for single sources. 

\begin{figure}[!ht]
    \centering
    \includegraphics[width = 0.4\textwidth, angle = 90]{firstexp.jpeg}
    \label{fig:4}
    \caption{Experimental set-up for the first experiment}
\end{figure}

To measure the effect of combining multiple sources together, the same method
of measurement is used, whilst having the two different types of sources
connected, either in parallel, or in series. 

Figures 5(a) and 5(b) show the physical set-up of the combined source circuits.

\begin{figure}[!ht]
    \centering
    \begin{subfigure}{0.5\textwidth}
        \centering
        \includegraphics[width = 0.6\linewidth, angle = 90]{series.jpeg}
        \label{fig:5a}
        \caption{Series sources circuit}
    \end{subfigure}%
    \begin{subfigure}{0.5\textwidth}
        \centering
        \includegraphics[width = 0.6\linewidth, angle = 90]{parallel.jpeg}
        \label{fig:5b}
        \caption{Parallel sources circuit}
    \end{subfigure}
    \label{fig:5}
    \caption{Experimental set-ups of the combined sources}
\end{figure}


\end{document}