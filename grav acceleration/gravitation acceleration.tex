\documentclass[a4paper]{article}
\usepackage{graphicx}
\usepackage[margin = 1in]{geometry}
\usepackage{ragged2e}
\usepackage{parskip}
\usepackage{amsmath}
\newcommand{\unit}[1]{~\mathrm{#1}}

\begin{document}
\begin{justify}

\section{Introduction}
\subsection{Research Question/ Problem Statement}
The objective of this lab is to analyze the effect of gravity during free fall
of a slat. The slat is dropped between an optical transmitter and receiver. This
is done to determine the position of the slat in terms of time in order to
calculate the gravitational acceleration.

\subsection{Background/ Theory}
Free fall refers to a situation when an object is only acted on by gravitational
force. In this experiment, air resistance is negligible.

\noindent
A rectilinear uniform motion can be used to describe free fall, relating the
position in terms time. Uniform motion paths can be described using the
following equation:
\begin{equation}
    x(t) = x_0 + v_x(t-t_0)
\end{equation} 
Where $v_x$ is defined as:
\begin{equation}
    v_x = \frac{x_2 - x_1}{t_2 - t_1}
\end{equation}
However, it is important to note that in free fall, the speed is not constant,
therefore the equation for speed becomes:
\begin{equation}
    v_x = \frac{dx}{dt}
\end{equation}
The instantaneous acceleration is defined by:
\begin{equation}
    a_x = \frac{dv_x}{dt}
\end{equation}
Considering air resistance as negligible, using equation 3 the following
equation is found:
\begin{gather}
    dx = v_x \cdot dt \\
    dx = (v_0 +a_x t) \cdot dt \nonumber
\end{gather}
By integrating, we get:
\begin{equation}
    x(t) = v_0 t + \frac{a_x t^2}{2}
\end{equation}
The gravitational acceleration varies based on where on the planet it is
measured. For example, it is smaller at the equator than the poles.

\noindent
When the position of the object is measured, the difference in time must be as
small as possible to get the most accurate results for the instantaneous
acceleration. Therefore, multiple points of measurement need to be used.

\noindent
In this experiment, the point in time is registered when the object has fallen a
certain distance, $50\unit{mm}$. Thus, we can calculate the instantaneous
acceleration and velocity. Notably, when calculating the instantaneous speed,
the difference in time, $\Delta t$, should approach 0.
\begin{equation}
    v_x = \lim_{\Delta t \rightarrow 0}\frac{\Delta x}{\Delta t}
\end{equation}
To increase precision when measuring speed, the velocity also needs to be
calculated for intermediate points in time. Having two successive measurement
points, the speed is considered accurate when taken exactly at the half way
point between the two measurements.
\begin{equation}
    t_{i+\frac{1}{2}} = \frac{t_{i+1} + t_i}{2}
\end{equation}
Therefore we calculate:
\begin{equation}
    v_{i+\frac{1}{2}} \approx \frac{\Delta x}{t_{i+1} - t_i}
\end{equation}
\section{Method \& Materials}
\subsection{Experimental Set-up}
Figure 1 shows the set-up for the experiment. The sensor gate, the alternating
stripes and the wire are shown. All the data collected by the software is
captured using PASCO's Capstone program.

ADD PICTURE
\subsection{Measuring instruments}
In this experiment, the PASCO smart gate, model PS-3225,  was used to measure the difference in
time, having a resolution of 2 ms (refr to the gate manual). PASCO's Capstone
software was used to collect the data, and a slate with 7 black strips 50 mm
apart was used as the free fall object.
\subsection{Method}
A slat with alternating stripes is dropped through the light sensor. The
resulting measurements are then used to create a table displaying time and
position. 

\noindent
This procedure is repeated twelve times, ensuring the data falls in the expected
range. All runs with erroneous data points are remeasured, to improve accuracy. 

\noindent
Due to the small mass and surface area of the slat, all external forces can be
ignored, and the resulting acceleration can be considered the gravitational acceleration.
\end{justify}
\end{document}